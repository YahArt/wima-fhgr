\documentclass[12pt, oneside]{article}
\usepackage{geometry}
\usepackage{url}
\urlstyle{same}
\usepackage[utf8]{inputenc}
\usepackage{setspace}
\usepackage{times}
\usepackage{multirow}
\usepackage[usegeometry]{typearea}
% Useful for checking layout
% \usepackage{showframe}
\usepackage{fancyhdr}
\usepackage{blindtext}
% Indenting the first sentence after section
\usepackage{indentfirst}
% Set language
\usepackage[german]{babel}
% Images
\usepackage{graphicx}
\graphicspath{ {./images/} }


\geometry{
 a4paper,
 left=30mm,
 top=25mm,
 right=25mm,
 bottom=20mm,
 footskip=15pt,
}

\setstretch{1.3} % Define Line Spacing
\renewcommand{\headrulewidth}{0pt} % Remove footer line

\pagestyle{fancy} % Allow for customizing header and footer
% Customize footer for page number location
\fancyhf{}
\fancyfoot{}
\fancyhead[R]{Yannick Hutter}
\fancyhead[L]{Wissensmanagementmodell}
\fancyfoot[R]{\thepage}

\begin{document}

\begin{titlepage}
	\begin{center}
		\Huge
		\textbf{Wissensmanagementmodell SOLVE}
		
		\vspace{0.5cm}
		\LARGE
		Erstellung eines Wissens-Newsletters zur Förderung von internem IT-Wissen in der Unternehmung SOLVE
		
		\vspace{1.0cm}
		\normalsize
		\textbf{Yannick Hutter}\\
		\textbf{Digital Business Management Klasse 18tz}\\
		\textbf{Talackerstrasse 8}\\
		\textbf{8887 Mels}\\
		\textbf{yannick.hutter@stud.fhgr.ch}\\
		\textbf{Mels, Juni 2022}\\
		
		\vspace{0.8cm}
		
		\begin{figure}[ht]
        	\includegraphics[width=10cm]{images/WIM_Titelblatt.png}
        	\centering
        \end{figure}
	\end{center}
\end{titlepage}

\clearpage
\section{Wissensmanagement Modell zur Erstellung eines Wissens-Newsletters}
\vfill
\begin{figure}[ht]
	\includegraphics[width=\textwidth]{images/WIM_Modell.png}
	\centering
\end{figure}
\vfill

\clearpage
\section{Themenwahl und Themenabgrenzung}
Die Firma Solve ist ein Engineering Unternehmen welches IT und Elektronik Dienstleistungen anbietet. In der Firma Solve hat jeder Mitarbeiter die Möglichkeit, Wissen zu einem bestimmten technologischen Thema was sein Interesse weckt, zu vertiefen. Das Model der vorliegenden Arbeit soll aufzeigen, wie mit der automatisierten Erstellung eines Wissens-Newsletters internes IT-Wissen gefördert und verbreitet werden kann. Die hierfür 5 notwendigen Schritte zur Erstellung eines Wissens-Newsletters werden nun anhand des Beispiels \textit{IoT} (Internet of Things) veranschaulicht. Der Solve Mitarbeiter Hubert interessiert sich für die Technologie IoT. In der ersten Phase \textbf{Aufarbeitung von Wissen} eignet sich Hubert daher das notwendige Gundwissen zu diesem Thema an. Dies kann zum Beispiel mithilfe eines Workshops realisiert werden. Anschliessend wird das gewonnene Wissen in der Phase \textbf{Anwendung von Wissen} in Form eines kleinen internen Projektes angewandt. Bei der Firma Solve kann dies zum Beispiel bei der Mitarbeit des Projektes Cafeteria I4.0 erfolgen (siehe \url{https://cafeteria-i40.ch/}). Danach werden die verschiedenen Learnings von Hubert aus der durchgeführten Mitarbeit des Cafeteria Projektes in der Phase \textbf{Zusammenfassung von Wissen} in Form einer Präsentation veranschaulicht. Die erstellte Präsentation trägt Hubert in der Phase \textbf{Vorstellen von Wissen} an der monatlichen internen Entwicklersitzung vor. Als letzten Schritt werden alle notwendigen Artefakte wie Präsentation, Quellcode etc. im SharePoint abgelegt. Hierbei achtet Hubert darauf, dass er die Artefakte mit entsprechenden Metainformationen anreichert. So definiert Hubert bei der Ablage die Metainformationen \#IoT sowie \#Wissens-Newsletter. Beim Hochladen der Dateien kommt nun der Automatisierungsprozess zum Tragen. Dieser erstellt automatisch aufgrund der hinterlegten Metainformationen pro Monat einen Wissens-Newsletter, welcher auch an die Mitarbeiter von Solve versendet wird. So können die Mitarbeiter jederzeit mitverfolgen, welches neues Wissen in welchen Kategorien von Monat zu Monat hinzugekommen ist. Dies stellt eine bessere \textit{Awareness} gegenüber dem fortlaufend steigenden Wissen innerhalb der gesamten Unternehmung sicher. Zudem können Mitarbeiter durch den Newsletter motiviert werden, sich einem eigenen Thema zu widmen, was zu einem Wechselwirkungseffekt führt, da dann noch mehr Wissen in einem weiteren Newsletter vorhanden ist und dieser wiederum mehr Mitarbeiter für bestimmte Themen inspirieren kann. Als \textbf{Erfolgsfaktoren} für das Modell sind sicherlich motivierte Mitarbeiter, welche sich mit neuen Technologien auseinandersetzen wollen zu nennen. Zudem sind auch Personen, welche ein grosses Interesse in einem spezifischen Technologiefeld besitzen (Beispiel IoT) sehr wertvoll, da hierdurch gezieltes Expertenwissen akummuliert werden kann. Dem Gegenüber stehen \textbf{Misserfolgsfaktoren} wie Mitarbeiter, welche kein Interesse an Technologien besitzen. Dieser Misserfolgsfaktor kann jedoch bereits bei der Rekrutierung eines Mitarbeiters überprüft werden. Ein weiterer Misserfolgsfaktor kann das Ignorieren des Wissens-Newsletters sein. Um dieser Thematik entgegenzuwirken, ist die gesamte Unternehmung Solve gefordert. Jeder Mitarbeiter soll seine Kollegen auf Wissen in behandelten Technologiefeldern aufmerksam machen und so womöglich ungeahntes Interesse wecken. 

\newpage
\section{Kritische Selbstreflexion}
Zu Beginn des Moduls Wissensmanagement war ich zuerst skeptisch gegenüber dem eigentlichen Nutzen. Jedoch habe ich durch die Entwicklung meines eigenen Wissensmodells festgestellt, dass Wissensmanagement ein sehr zentraler und wichtiger Aspekt eines Unternehmens darstellt. Am Anfang wusste ich was für einzelne Schritte die Firma Solve durchführt, um Wissen zu generieren, zu festigen und zu verteilen. Jedoch bin ich mir erst durch die Erstellung eines Models im Klaren geworden, wie wichtig die einzelnen Strukturen und Schritte wirklich sind. Auch sind mir bei der Erstellung wichtige Wechselwirkungen aufgefallen, welche von hoher Relevanz sind. Bei der Erstellung des Models musste ich besonders darauf Achten, mich nicht im Detail zu verlieren, sondern den Blick auf das grosse Ganze beizubehalten. Auch war es schwierig die gesamten Konstrukte möglichst einfach und verständlich abzubilden, sodass eine schnelle Interpretation möglich ist. Auch hatte ich zu Beginn Probleme mit der Erstellung der Legende. Diese war in der ersten Version der Arbeit zu allgemein gehalten und beschrieb die relevanten Grafiken unzureichend. Das erhaltene Feedback hat mir geholfen das Modell neu zu definieren, auf die wesentlichsten Punkte zu reduzieren sowie eine spezifischere Legendenbeschreibung zu erstellen. Die Identifikation von Erfolgs- sowie Misserfolgsfaktoren meines Modells haben mir geholfen wichtige Gefahren aber auch Chancen zu identifizieren und mir auch Gedanken über geeignete Strategien zur Förderung von Erfolgs-, sowie zur Verminderung der Misserfolgsfaktoren zu überlegen. Abschliessend kann ich sagen, dass mir die Entwicklung eines Wissensmodells enorm dabei geholfen, Schwachstellen bzw. Verbesserungspotenzial in Bezug auf das Wissensmanagement der Firma Solve zu erkennen. So konnte ich enormes Potenzial mit Hilfe von SharePoint Automatisierung in Kombination mit der Erstellung von Wissens-Newslettern entdecken und betrachte nun die Thematik Wissensmanagement von einem ganz neuen und positiven Blickwinkel.
\end{document}